%% Preambel
\documentclass[conference,compsoc,final,a4paper,12pt]{IEEEtran}
\usepackage[utf8]{inputenx}

\newcommand{\autoren}[0]{Ahmed Salame, Ida Vanessa Gouleu Mokam}

\newcommand{\dokumententitel}[0]{Websockets mit Grails}

\input{preambel} % Weitere Einstellungen aus einer anderen Datei lesen

% Eigentliches Dokument beginnt hier
% ----------------------------------------------------------------------------------------------------------

% Kurze Zusammenfassung des Dokuments
\begin{abstract}
Dieses Paper soll aufzeigen, wie Websockets mit Grails realisiert werden kann. Zunächst wird auf einige Grundlagen näher eingegangen, um dem Leser ein Besseres Verständnis für das Thema zu vermitteln. 

--TODO--
\end{abstract}

\tableofcontents

\section{Einleitung}
 
\section{Forschungsmethode}

Beim Recherchieren wurde unter anderem Google Books\footnote{https://books.google.de/} verwendet, um passende Literatur zu finden. Des weiteren wurde die Homepage The Groovy Project herangezogen, um die Sprache Groovy näher zu beleuchten.
Zudem wurde nach passenden Quellen aus dem Internet gesucht, um bestimmte Themen näher zu erläutern. Zudem sind diese auch für jeden zugänglich. 

\section{State of the art}

Hier wird der Aktuelle Stand aufgezeigt


\section{Einleitung}

Hier wird es eine Einleitung zum Thema geben. 

\section{Forschungsmethode}

Hier wird beschrieben, wie genau vorgegangen bei der literatur suche des Papers

\section{State of the art}

Hier wird der Aktuelle Stand aufgezeigt

\section{Grundlagen}

Hier sollen Grundlagen geklärt werden

\subsection{Websockets}

Eine allgemeine Erklärung was Websockets sind

\subsection{Groovy on Grails}

Hier soll erklärt werden, dass Goovy ist ( mit bezug zu java) und was Groovy on Grails ist.

\subsection{Spring}

Hier soll es eine kurze erklärung geben, was Spring genau ist und wozu es genutzt wird. 

\section{Annotationen}

Hier sollen die Annotationen, die beim Websocket genutzt werden, erklärt werden.

\section{Implementierung}

Hier soll erklärt werden, was genau getan werden muss, um es zu nutzen.

\section{Fazit}

\input{ida_text}

\section{Fazit}

%--------------------------

\lstlistoflistings

\listoffigures

\section*{Abkürzungen}
\addcontentsline{toc}{section}{Abkürzungen}

\begin{acronym}
\acro{POJO}{Plain Old Java Object}
\acro{GSP}{Groovy Server Pages}
\acro{JTA}{Java Transaction API}
\acro{JSR}{Java Specification Request}
\acro{JVM}{Java Virtual Machine}
\acro{URI}{Uniform Resource Identifier}
\end{acronym}

% Literaturverzeichnis
\addcontentsline{toc}{section}{Literatur}
\printbibliography

\end{document}